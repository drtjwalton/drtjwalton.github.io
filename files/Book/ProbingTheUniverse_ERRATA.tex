\documentclass[12pt]{GeomPhysBOOK}

\usepackage[a4paper,left=1.8cm,right=1.8cm,top=1.5cm,bottom=2.3cm]{geometry}

\newcommand\correct[1]{{\color{red!80!black}\bf #1}}
\newcommand\listrule{\vspace{0.2cm}\hrule height 0.3pt}

\clearpairofpagestyles
\lofoot{\underline{Version date}: 20/12/2025} \cofoot{} \rofoot{Page \pagemark} 
\lefoot{\underline{Version date}: 20/12/2025} \cefoot{} \refoot{Page \pagemark} 
\setkomafont{pagenumber}{\normalsize} 
\pagestyle{scrheadings}
\setheadsepline{0pt}
\setfootsepline{0pt}


%%%%%%%%%%%%%%%%%%%%%%%%%%%%%%%%%%%%%%%%%%%%%%%%%%%%%%%%%%%%%%%%%%%%%%%%%%%%%%%%%%%%%%
\begin{document}
%%%%%%%%%%%%%%%%%%%%%%%%%%%%%%%%%%%%%%%%%%%%%%%%%%%%%%%%%%%%%%%%%%%%%%%%%%%%%%%%%%%%%%

\rule{0.999\textwidth}{0.05cm}\vspace{0.2cm}
\begin{center}
	\begin{minipage}{0.18\textwidth}
		{\bf \large Errata for:} 		
	\end{minipage}\;
	\begin{minipage}[t]{0.8\textwidth}
		\centering
		{\Large\bf ``Probing the Universe: A Geometrical View for} \\[0.2cm] {\Large\bf Observers of Spacetime Physics''} \\[0.4cm]
		{\large Robin W. Tucker and Timothy J. Walton (Springer: 2026)}
	\end{minipage}
\end{center}
\rule{0.47\textwidth}{0.05cm}\parbox{1cm}{\vspace{0.1cm}\centering\normalsize \OrnamentDiamondSolid}\rule{0.47\textwidth}{0.05cm}\vspace{0.7cm}

A list of typographical errors and correction for the book will be kept here. Corrections are emphasised in \correct{red} text. \\[0.3cm]
Please email \href{mailto:TWalton@lancashire.ac.uk}{TWalton@lancashire.ac.uk} if you find any typographical errors not already listed.\\[0.1cm]

%%%%%%%% ERRATA
\begin{enumerate}[leftmargin=1.9cm,labelwidth=1.4cm,itemsep=0.5cm]
	%%%%%%%%%%%%%%%%%%%%%%%%%%%%%%%%%%%%%%%%%%%
	\listrule\vspace{0.24cm}
	\item[\underline{page 376}:] 	
		The last displayed equation on this page should read:
		\begin{align*}
			\EINFM{c} &\,=\, \mathcal{R}\star e_{c} - 2(i_{a}P_{c})\star e^{a} + \correct{2}(i_{c}i_{a}i_{b}\,\GLD T^{b})\, \star e^{a} \\[0.2cm]
				&\,=\, \mathcal{R}\star e_{c} - 2\star P_{c} - 2i_{a}(i_{c}i_{b}\,\GLD T^{b})\, \star e^{a} 
		\end{align*}
	%%%%%%%%%%%%%%%%%%%%%%%%%%%%%%%%%%%%%%%%%%%
	\listrule
	\item[\underline{page 383}:] 	
		The top displayed equation on this page should read:
		\begin{align*}
			\psi_{2} \,=\, \varye{\correct{a}}\w e^{b} \w \star R_{ab}. 
		\end{align*}
		The displayed equation between (20.17) and (20.18) should read:
		\begin{align*}
			\GLD\correct{\star} e^{ab} &\,=\, \frac{1}{2}\,\epsilon^{ab}{}_{cd}\,\GLD e^{cd} \,=\, \frac{1}{2}\,\epsilon^{ab}{}_{cd}\,\left(T^{c} \w e^{d} - e^{c}\w T^{d} \right) \\[0.2cm]
				&\,=\, \epsilon^{ab}{}_{cd}\,T^{c} \w e^{d} \,=\, T_{c}\w \star e^{abc}
		\end{align*}
	%%%%%%%%%%%%%%%%%%%%%%%%%%%%%%%%%%%%%%%%%%%
	\listrule
	\item[\underline{page 384}:] 	
		The line of text above equation (20.22) should read:
		\begin{quote}
			``In this case the $3$-forms $\{\EINEHFM{c}\equiv\correct{-\frac{1}{2}}\EINFM{c}\}$ define$\ldots$''\\
		\end{quote}
	%%%%%%%%%%%%%%%%%%%%%%%%%%%%%%%%%%%%%%%%%%%
	\listrule
	\item[\underline{page 385}:] 	
		Equation (20.24) should read:
		\begin{align}\tag{20.24}
			\begin{split}
				\Lambda_{\text{TOTAL}}(e,\omega,A,\phi,\phibar\,) &\,\equiv\, \correct{\frac{1}{2\kappa}}\,R_{ab}(\omega) \w \star e^{ab} + \kappa_{A}dA \w \star dA  \\[0.2cm]
					& \hspace{2cm} + \kappa_{1}\text{Re}\left( \overline{\GLD_{\text{U}(1)}\phi} \w \star \GLD_{\text{U}(1)}\phi \right) \\[0.2cm]
					& \hspace{2cm} + \kappa_{2}\,\phi\,\phibar\,\star 1 + \kappa_{3}\,R_{ab}(\omega) \w R^{ab}(\omega)
			\end{split}		
		\end{align}
	%%%%%%%%%%%%%%%%%%%%%%%%%%%%%%%%%%%%%%%%%%%
	\listrule
	\item[\underline{page 386}:]
		Equation (20.30) should read:
		\begin{align}\tag{20.30}
			\ACTION{EH}[e,\omega] \,\equiv\, \correct{\frac{1}{2\kappa}}\int_{\M}R_{ab}(\omega) \w \star (\,e^{a} \w e^{b}\,), 
		\end{align} 
		and the displayed equation below becomes:
		\begin{align*}
			\EINFM{c}^{(\text{EH})} \,\equiv\, \correct{-\frac{1}{2}}\,R_{ab} \w i_{c}\star (\,e^{a} \w e^{c}\,) &\,=\, 0, \quad a,b,c=0,1,2,3, \; e^{a}(X_{c}) \,=\, \deltaT^{a}_{c}   \\[0.2cm]
			T^{c} &\,=\, 0.
		\end{align*}
	%%%%%%%%%%%%%%%%%%%%%%%%%%%%%%%%%%%%%%%%%%%
	\listrule
	\item[\underline{page 388}:]
		Equation (20.31) should read:
		\begin{align}\tag{20.31}
			\EIN \,\equiv\, (\star^{-1}\EINFM{c}^{\correct{(\text{EH})}}) \tensor e^{c} \,=\, \kappa\,(\star^{-1}\tau_{c}^{(\text{SOURCE})}) \tensor e^{c}. 
		\end{align} 
	%%%%%%%%%%%%%%%%%%%%%%%%%%%%%%%%%%%%%%%%%%%
	\listrule
	\item[\underline{page 392}:]
		The first equation of section~20.4 should read:
		\begin{align}\tag{20.30}
			\ACTION{EH}[e,\omega] \,\equiv\, \correct{\frac{1}{2\kappa}}\int_{\M}R_{ab}(\omega) \w \star (\,e^{a} \w e^{b}\,), 
		\end{align} 
		and the second equation becomes:
		\begin{align*}
			\correct{\ACTION{SOURCE}}[e,A,\phi,\phibar\,] \,\equiv\, \kappa_{A}\int_{\M} \Lambda_{2}(e,A) &+ \kappa_{1}\int_{\M}\Lambda_{1}(e,A,\phi,\phibar\,) \\[0.2cm]
				&+ \mu^{2}\int_{\M}\Lambda_{3}(e,\phi,\phibar\,) \qquad (\kappa_{1},\mu\in\RR)
		\end{align*} 
	%%%%%%%%%%%%%%%%%%%%%%%%%%%%%%%%%%%%%%%%%%%
	\listrule
	\item[\underline{page 395}:]
		The displayed equation immediately above equation (20.37) should read:
		\begin{align*}
			\whh{\delta}\left(\df \correct{\ACTION{EH}}[e,\omega] \;\correct{+}\; \ACTION{SOURCE}[e,A,\phi,\phibar]\right) \,\simeq\, 0,
		\end{align*}
	%%%%%%%%%%%%%%%%%%%%%%%%%%%%%%%%%%%%%%%%%%%
	\listrule
	\item[\underline{page 396}:]
		Equation (20.39) should read:
		\begin{align}\tag{20.39}
			\frac{\EINFM{c}^{\correct{(\text{EH})}}}{\kappa} \,=\, \tau_{c}
		\end{align} 
		and the text immediately after this displayed equation becomes:
		\begin{quote}
			``where $\EINFM{c}^{\correct{(\text{EH})}}\equiv -\frac{1}{2}R_{ab} \w i_{c}\star e^{ab} \ldots$''.
		\end{quote}
		The text in the first paragraph immediately below the last displayed equation of this page should be:
		\begin{quote}
			``Note $\PHYSDIM{\tau_{c}}=\MASS\PHYSDIM{c_{0}^{2}}$. Since the $\mathring{\omega}_{ab}$ partial variations in $\ACTION{EH}[e,\omega]$ {\it and} $\ACTION{SOURCE}[e,A,\phi,\phibar]$ yield a set of zero-torsion connection $1$-forms (see section above), if these are assumed to be metric-compatible with respect to the metric tensor $\g=\etaT_{ab}\,e^{a}\tensor e^{b}$ then (\ref{PHI_15_01_24}), (\ref{MAX1_15_01_24}), (\ref{Gc_07_01_24}) constitute the necessary conditions for the functional \correct{$\ACTION{EH}[e,\omega]+\ACTION{SOURCE}[e,A,\phi,\phibar]$} to be extremal and (\ref{XXX_15_01_24}) yields a {\it Levi-Civita Einstein tensor field equation with sources}. ''\\
		\end{quote}
	%%%%%%%%%%%%%%%%%%%%%%%%%%%%%%%%%%%%%%%%%%%
	\listrule
	\item[\underline{page 402}:] 	
		The penultimate displayed equation on this page should read:
		\begin{align*}
			\texttt{T}^{(\text{MAX})}(X_{0},X_{0}) \,=\, (\star \tau^{(\text{MAX})}_{c})(X_{0})\,\correct{\deltaT^{c}_{0}} \,=\, i_{0} (\star \tau^{(\text{MAX})}_{0})
		\end{align*}
	%%%%%%%%%%%%%%%%%%%%%%%%%%%%%%%%%%%%%%%%%%%
	% \listrule
	% \item[\underline{page 402}:] 
	% %%%%%%%%%%%%%%%%%%%%%%%%%%%%%%%%%%%%%%%%%%%
	% \listrule
	% \item[\underline{page 402}:] 
	% %%%%%%%%%%%%%%%%%%%%%%%%%%%%%%%%%%%%%%%%%%%
	% \listrule
	% \item[\underline{page 402}:] 
	% %%%%%%%%%%%%%%%%%%%%%%%%%%%%%%%%%%%%%%%%%%%
	% \listrule
	% \item[\underline{page 402}:] 
	%%%%%%%%%%%%%%%%%%%%%%%%%%%%%%%%%%%%%%%%%%%%
	\listrule
\end{enumerate}
%%%%%%%% END OF ERRATA

%% PREFACE %%%%%%%%%%%%%%%%%%%%%%%%%%%%%%%%%%%%%%%%%%%%%%%%%%%%%%%%%%%%%%%%%%%%%%%%%%%

%%%%%%%%%%%%%%%%%%%%%%%%%%%%%%%%%%%%%%%%%%%%%%%%%%%%%%%%%%%%%%%%%%%%%%%%%%%%%%%%%%%%%
\end{document}
%%%%%%%%%%%%%%%%%%%%%%%%%%%%%%%%%%%%%%%%%%%%%%%%%%%%%%%%%%%%%%%%%%%%%%%%%%%%%%%%%%%%%
